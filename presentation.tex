% !TEX TS-program = XeLaTeX
%!TEX encoding = UTF-8 Unicode
%-------------------------------------------------------
% Author: Christian Häusler based on the work from Thomas Bruderer
% Email: christian.haeusler@piratenpartei.ch
%-------------------------------------------------------

% aspectratio
%	1610	≙	16:10
%	169	≙	16:9
%	149	≙	14:9
%	54	≙	5:4
%	43	≙	4.3
%	32	≙	3:2
%
\documentclass[aspectratio=1610, compress]{beamer}
% -------------------------------------------
% Use the following three lines instead of the above 
% to create a handout
% -------------------------------------------
%\documentclass{scrartcl}
%\usepackage{ppsdocument}
%\usepackage[noxcolor, hyperref]{beamerarticle}

% -------------------------------------------
% Load the pps beamer theme
% Possible options are:
%	miniframes (use the outer theme miniframes)
%	structured (use the outer theme sidebar)
%		right (display the sidebar on the right side)
%		left (display the sidebar on the left side: default)
%		width=<dimension> (width of the sidebar)
%		heifht=<dimension> (height of the frame title)
% -------------------------------------------
\usetheme{pps}

%\setbeamertemplate{section in head/foot shaded}[default][70]
%\setbeamertemplate{subsection in head/foot shaded}[default][70]
 
% -------------------------------------------
% Uncomment one of the following lines for a section
% specific document
% -------------------------------------------
%\usepackage{ppsag}
%\usepackage{ppsbb}
%\usepackage{ppsbe}
%\usepackage{ppsvd}
%\usepackage{ppszh}

% -------------------------------------------
% Change from here
% -------------------------------------------
\title[Short Title]{Title in full length}
%\subtitle<presentation>[Short subtitle]{Subtitle in full length}
\subtitle{Subtitle in full length}
\author[Short Name]{Author Name in full length}
\date[\today]{\today}
\subject{The subject is inserted in the meta data of the pdf and helps search engines to find the document}
\keywords{Keywords are inserted in the meta data of the pdf and helps search engines to find the document}

\begin{document}
% -------------------------------------------
% Set the language will also change the logo
% Possible options are:
% 	english
% 	french
% 	italian
% 	ngerman
% -------------------------------------------
\selectlanguage{ngerman}

\frame[plain]{\maketitle}

%\section<presentation>{This section exists only in the presentation modes}
%\section<article>{This section exists only in the article mode}
%\section[Summary]{Summary of Main Results}
%\hyperlink{somewhere}{\beamerbutton{Go somewhere}}

% Toc only shown in the handout
\frame<article>[plain]{\tableofcontents\newpage}

\part[Part 1]{Full title of Part 1}

\frame[plain]{\partpage}

% Toc for part 1 only shown in presentation
\frame<presentation>[plain]{\frametitle{Inhaltsverzeichnis}\tableofcontents}

\section{Abschnitt Nr.1} 
\begin{frame}\frametitle{Titel} 
	Die einzelnen Frames sollte einen Titel haben 
\end{frame}

\subsection{Unterabschnitt Nr.1.1  }
\begin{frame}\frametitle{Testtitel}
Denn ohne Titel fehlt ihnen was
\end{frame}

\section{Abschnitt Nr.2} 
\subsection{Listen I}
\begin{frame}\frametitle{Aufz\"ahlung}
\begin{itemize}
\item Einf\"uhrungskurs in \LaTeX  
\item Kurs 2  
\item Seminararbeiten und Pr\"asentationen mit \LaTeX 
\item Die Beamerclass 
\end{itemize} 
\end{frame}

\begin{frame}\frametitle{Aufz\"ahlung mit Pausen}
\begin{itemize}
\item  Einf\"uhrungskurs in \LaTeX \pause 
\item  Kurs 2 \pause 
\item  Seminararbeiten und Pr\"asentationen mit \LaTeX \pause 
\item  Die Beamerclass
\end{itemize} 
\end{frame}

\backgroundviolet

\subsection{Listen II}
\begin{frame}\frametitle{Numerierte Liste}
\begin{enumerate}
\item  Einf\"uhrungskurs in \LaTeX 
\item  Kurs 2
\item  Seminararbeiten und Pr\"asentationen mit \LaTeX 
\item  Die Beamerclass
\end{enumerate}
\end{frame}
\begin{frame}\frametitle{Numerierte Liste mit Pausen}
\begin{enumerate}
\item  Einf\"uhrungskurs in \LaTeX \pause 
\item  Kurs 2 \pause 
\item  Seminararbeiten und Pr\"asentationen mit \LaTeX \pause 
\item  Die Beamerclass
\end{enumerate}
\end{frame}

\section{Abschnitt Nr.3} 
\subsection{Tabellen}
\begin{frame}
\frametitle{Tabellen}
\begin{tabular}{|c|c|c|}
\hline
\textbf{Zeitpunkt} & \textbf{Kursleiter} & \textbf{Titel} \\
\hline
WS 04/05 & Sascha Frank &  Erste Schritte mit \LaTeX  \\
\hline
SS 05 & Sascha Frank & \LaTeX \ Kursreihe \\
\hline
\end{tabular}
\end{frame}

\backgroundgrey

\begin{frame}
\frametitle{Tabellen mit Pause}
\begin{tabular}{c c c}
A & B & C \\ 
\pause 
1 & 2 & 3 \\  
\pause 
A & B & C \\ 
\end{tabular} 
\end{frame}

\backgrounddefault

\part[Part 2]{Full title of Part 2}

\frame[plain]{\partpage}

\section{Abschnitt Nr.4}
\subsection{Bl\"ocke}
\begin{frame}\frametitle{Bl\"ocke}

\begin{block}{Blocktitel}
Blocktext 
\end{block}

\begin{exampleblock}{Blocktitel}
Blocktext 
\end{exampleblock}


\begin{alertblock}{Blocktitel}
Blocktext 
\end{alertblock}
\end{frame}

\begin{frame}
   \frametitle{A Theorem on Infinite Sets}
   \begin{theorem}<1->
     There exists an infinite set.
   \end{theorem}
   \begin{proof}<2->
     This follows from the axiom of infinity.
\end{proof}
   \begin{example}<3->[Natural Numbers]
     The set of natural numbers is infinite.
   \end{example}
 \end{frame}

% Nothing after this line will show up in the toc
\appendix

\section[Quellen]{Referezen}
\begin{frame}\frametitle{Quellen \& Literatur}

\begin{thebibliography}{Beamerdokumentation}
\bibitem[Beamerpaket]{paket} \emph{Beamer Paket} \\ 
\text{http://latex-beamer.sourceforge.net/}
\bibitem[Beamerdokumentation]{doku} \emph{User's Guide to the Beamer} 
\bibitem[Dante]{dante} \emph{DANTE e.V.} \text{http://www.dante.de}   
\end{thebibliography}


\end{frame}

\end{document}