%!TEX encoding = UTF-8 Unicode%-------------------------------------------------------
% Author: Christian H�usler based on the work from Thomas Bruderer
% Email: christian.haeusler@piratenpartei.ch
%-------------------------------------------------------
\documentclass{beamer}
\usepackage{ppsbeamer}

% -------------------------------------------
% Uncomment one of the following lines for a section
% specific document
% -------------------------------------------
%\usepackage{ppsag}
%\usepackage{ppsbb}
%\usepackage{ppsbe}
%\usepackage{ppsvd}
%\usepackage{ppszh}

% -------------------------------------------
% Change from here
% -------------------------------------------
\title[Title on Document Footer]{Full Title on Titlepage}
\author[Name on Document Footer]{Ful Author Name on Titlepage}
\date[\today]{\today}
\subject{Subject}

\begin{document}
% -------------------------------------------
% Set the language will also change the logo
% Possible options are:
% 	english
% 	french
% 	italian
% 	ngerman
% -------------------------------------------
\selectlanguage{ngerman}

\begin{frame}
\titlepage
\end{frame}

\begin{frame}
\frametitle{Inhaltsverzeichnis}\tableofcontents
\end{frame}


\section{Abschnitt Nr.1} 
\begin{frame}\frametitle{Titel} 
Die einzelnen Frames sollte einen Titel haben 
\end{frame}

\subsection{Unterabschnitt Nr.1.1  }
\begin{frame}\frametitle{Testtitel}
Denn ohne Titel fehlt ihnen was
\end{frame}


\section{Abschnitt Nr.2} 
\subsection{Listen I}
\begin{frame}\frametitle{Aufz\"ahlung}
\begin{itemize}
\item Einf\"uhrungskurs in \LaTeX  
\item Kurs 2  
\item Seminararbeiten und Pr\"asentationen mit \LaTeX 
\item Die Beamerclass 
\end{itemize} 
\end{frame}

\begin{frame}\frametitle{Aufz\"ahlung mit Pausen}
\begin{itemize}
\item  Einf\"uhrungskurs in \LaTeX \pause 
\item  Kurs 2 \pause 
\item  Seminararbeiten und Pr\"asentationen mit \LaTeX \pause 
\item  Die Beamerclass
\end{itemize} 
\end{frame}

\subsection{Listen II}
\begin{frame}\frametitle{Numerierte Liste}
\begin{enumerate}
\item  Einf\"uhrungskurs in \LaTeX 
\item  Kurs 2
\item  Seminararbeiten und Pr\"asentationen mit \LaTeX 
\item  Die Beamerclass
\end{enumerate}
\end{frame}
\begin{frame}\frametitle{Numerierte Liste mit Pausen}
\begin{enumerate}
\item  Einf\"uhrungskurs in \LaTeX \pause 
\item  Kurs 2 \pause 
\item  Seminararbeiten und Pr\"asentationen mit \LaTeX \pause 
\item  Die Beamerclass
\end{enumerate}
\end{frame}

\section{Abschnitt Nr.3} 
\subsection{Tabellen}
\begin{frame}
\frametitle{Tabellen}
\begin{tabular}{|c|c|c|}
\hline
\textbf{Zeitpunkt} & \textbf{Kursleiter} & \textbf{Titel} \\
\hline
WS 04/05 & Sascha Frank &  Erste Schritte mit \LaTeX  \\
\hline
SS 05 & Sascha Frank & \LaTeX \ Kursreihe \\
\hline
\end{tabular}
\end{frame}


\begin{frame}
\frametitle{Tabellen mit Pause}
\begin{tabular}{c c c}
A & B & C \\ 
\pause 
1 & 2 & 3 \\  
\pause 
A & B & C \\ 
\end{tabular} 
\end{frame}


\section{Abschnitt Nr.4}
\subsection{Bl\"ocke}
\begin{frame}\frametitle{Bl\"ocke}

\begin{block}{Blocktitel}
Blocktext 
\end{block}

\begin{exampleblock}{Blocktitel}
Blocktext 
\end{exampleblock}


\begin{alertblock}{Blocktitel}
Blocktext 
\end{alertblock}
\end{frame}

\section[Quellen]{Referezen}
\begin{frame}\frametitle{Quellen \& Literatur}

\begin{thebibliography}{9}
\bibitem[Beamerpaket]{paket} \emph{Beamer Paket} \\ 
\text{http://latex-beamer.sourceforge.net/}
\bibitem[Beamerdokumentation]{doku} \emph{User's Guide to the Beamer} 
\bibitem[Dante]{dante} \emph{DANTE e.V.} \text{http://www.dante.de}   
\end{thebibliography}


\end{frame}

\end{document}